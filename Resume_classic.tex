%----------------------------------------------------------------------------------------
%	DOCUMENT CONFIGURATION
%----------------------------------------------------------------------------------------
\documentclass[11pt]{article}           % Main text size is 11pt

% Page margin setup
\usepackage[a4paper,                    % Paper size (use letterpaper for US resumes)
            top=1.27cm,                 % Top margin
            bottom=1.27cm,              % Bottom margin
            left=1.27cm,                % Left margin
            right=1.27cm                % Right margin
           ]{geometry}

%----------------------------------------------------------------------------------------
%	FONTS AND ENCODING
%----------------------------------------------------------------------------------------
% Set font to Helvetica (Sans-Serif) for a modern, clean look
\usepackage[scaled]{helvet}             % Helvetica font package (scaled)
\renewcommand\familydefault{\sfdefault} % Use sans-serif as default font family
\usepackage[T1]{fontenc}                % Output font encoding (important for special chars)
\usepackage[utf8]{inputenc}             % Input encoding (UTF-8)
% \usepackage{fontspec}
% \setmainfont{Calibri} 
% \usepackage{XCharter}                 % comment this line out if you want to use the default LaTeX font Computer Modern

%----------------------------------------------------------------------------------------
%	PACKAGES
%----------------------------------------------------------------------------------------
\usepackage{xcolor}                     % Required for custom colors
\usepackage[explicit]{titlesec}         % Allows advanced section title formatting
\usepackage{graphicx}                   % Enables image inclusion
\usepackage{enumitem}                   % Customizes lists (bullet points)
\usepackage[hidelinks]{hyperref}        % Enables clickable links without ugly borders
\usepackage{amsmath}                    % Mathematical symbols (if needed)
%----------------------------------------------------------------------------------------
%	GLOBAL PAGE STYLES
%----------------------------------------------------------------------------------------
\raggedright                            % Disable text justification (better for columns/CVs)
\pagestyle{empty}                       % Disable page numbering

% Ensure PDF output is machine-readable (ATS friendly)
\input{glyphtounicode}
\pdfgentounicode=1

%----------------------------------------------------------------------------------------
%	SECTION FORMATTING
%----------------------------------------------------------------------------------------
% Format the section headers
\titleformat{\section}
  {\bfseries\large}           % Font styling: Bold and Large
  {}                          % Label: Empty (removes numbering like "1.")
  {0pt}                       % Separation: No space between label and title
  {
    \MakeUppercase{#1}        % THE TITLE: The actual section name (e.g., "EXPERIENCE")
  }                                 
  [\vspace{1pt}\titlerule]    % After-code: adds a horizontal line with spacing

% Adjust spacing around sections: {Command}{Left Indent}{Space Before}{Space After}
\titlespacing*{\section}
  {0pt}   % Left Indent: 0pt keeps it aligned with the left margin
  {7pt}   % Space Before: The vertical gap above the colored box
  {4pt}   % Space After: The vertical gap between the colored box and the text below
%----------------------------------------------------------------------------------------
%	LIST/BULLET POINT FORMATTING
%----------------------------------------------------------------------------------------
% Redefine the symbol for the first level of itemized lists (Change the bullet point symbol)
\renewcommand\labelitemi{%
  % Enter math mode ($...$) to use vertical centering tools
  $%
    \vcenter{%             % Vertically align the content inside to the math axis (center of line)
      \hbox{%              % Create a horizontal box to hold the text/symbol
        \small             % Make the bullet slightly smaller than the text for a cleaner look
        $\bullet$%         % The actual bullet symbol
      }%
    }%
  $%
}

% Global settings for itemize lists (tighten spacing). Requires: \usepackage{enumitem}
\setlist[itemize]{
  itemsep=-2pt,    % Vertical space BETWEEN items. Negative value pulls items closer together (compact).
  leftmargin=25pt, % Horizontal indent from the left edge of the text area.
  topsep=3pt       % Vertical space ABOVE the first item (separation from the paragraph above).
}

%----------------------------------------------------------------------------------------
%	MAIN DOCUMENT
%----------------------------------------------------------------------------------------
\begin{document}

% --- HEADER SECTION ---
% This layout is a single column, centered stack (Name on top, links below).
\noindent % Prevents paragraph indentation so the centering is exact.

% 1. NAME
% \centerline forces the content to be centered on a single line.
% Note: \centerline is rigid; if the text is too long, it will run off the page instead of wrapping.
\centerline{\Huge Muhammad Al-Khwarizmi}


\vspace{5pt} % Adds a small vertical buffer between Name and Contact Info

% 2. CONTACT INFO
% This line combines Email, Portfolio, and GitHub with pipes (|) as separators.
\centerline{%
    \href{mailto:scholar@houseofwisdom.iq}{scholar@houseofwisdom.iq} $|$ % Clickable Email
    \href{https://en.wikipedia.org/wiki/Al-Khwarizmi}{wikipedia.org/wiki/Al-Khwarizmi} $|$   % Clickable Portfolio
    \href{https://github.com/father-of-algebra}{github.com/father-of-algebra} $|$ % Clickable GitHub
    +964 000 12345 % Phone number 
}


\vspace{4pt} % Adjust space between the Header and the start of the Resume body
% \hrule % Uncomment if you want a horizontal line after the header for separation

% --- CONTENT SECTIONS ---
% These load external .tex files. Ensure these files exist in a "Sections" folder.
\section*{Objective}
{Dedicated polymath and Chief Scholar with a strong foundation in mathematics, astronomy, and geography, seeking to leverage extensive experience in manuscript synthesis to advance scientific knowledge and cultural exchange at the House of Wisdom, with the goal of developing foundational algebraic frameworks and cutting-edge astronomical tables.}
\section*{Education}
\textbf{Scholar of Mathematics, Astronomy, and Geography} $|$ \textbf{House of Wisdom (Bayt al-Hikmah)} \hfill \textbf{approx. 820 AD} \\
\textit{Focus: Studied and synthesized ancient Sanskrit, Persian, and Greek manuscripts under the direct patronage of Caliph Al-Ma'mun}
\section*{Experience}
\textbf{Chief Astronomer and Librarian (Lead Scholar)} $|$ \textbf{House of Wisdom, Baghdad} \hfill \textbf{820 AD -- 850 AD}
\begin{itemize}
    \item Directed a massive translation movement, rescuing and expanding upon classical works from antiquity.
    \item Advised Caliph Al-Ma'mun on scientific initiatives, successfully conducting complex astronomical observations.
    \item Managed one of the world's most robust and historically significant repositories of scholarly manuscripts.
\end{itemize}
\section*{Projects}
\textbf{Al-Kitab al-Mukhtasar fi Hisab al-Jabr wal-Muqabala} 
\begin{itemize}
    \item Authored the foundational text of algebra, pioneering systematic solutions for linear and quadratic equations.
    \item Coined the term "algebra" (derived from \textit{al-jabr}, meaning "completion" or "restoration").
\end{itemize}
\vspace{4pt}

\textbf{Hindu-Arabic Numeral Integration (Algorism)}
\begin{itemize}
    \item Developed a framework to adopt the decimal positional number system from Indian mathematics, enabling high-throughput calculation models.
    \item Simulated mathematical operations including addition, subtraction, multiplication, and division, proving extreme performance gains over Roman and Greek numeral systems.
    \item Evaluated systemic efficiency, resulting in my Latinized name ("Algoritmi") becoming the namesake for the modern concept of an "algorithm".
\end{itemize}
\vspace{4pt}

\textbf{Zij al-Sindhind (Astronomical Tables)} 
\begin{itemize}
    \item Compiled advanced tables tracking the movements of the sun, moon, and the five known planets.
    \item Introduced pivotal Indian trigonometric concepts (such as the sine function) into Islamic astronomy.
\end{itemize}
\vspace{4pt}

\textbf{Kitab Surat al-Ard (Book of the Description of the Earth)} 
\begin{itemize}   
    \item Corrected Ptolemy's geographical works, providing updated map coordinates for 2,402 cities and landmarks across the known world.
\end{itemize}
\section*{Technical Skills}
\begin{description}[
    leftmargin=3.5cm,           % Aligns the start of the text
    labelwidth=3.3cm,           % Width of the category name
    labelsep=0.2cm,             % Space between category and skills
    style=multiline,            % Allows category name to wrap if needed
    font=\bfseries,             % Category name in bold
    itemsep=-1pt                % Vertical space between rows
]
    \item[Mathematics] Algebra (Founding Father), Algorithms, Arithmetic, Hindu-Arabic Numerals, Trigonometry
    \item[Sciences] Observational Astronomy, Geography, Cartography, Horology
    \item[Languages] Arabic (Fluent/Scholarly), Persian (Native), Greek (Translation), Sanskrit (Translation)
\end{description}
\section{Interests}
\noindent
Navigational Instruments (Astrolabes, Sundials) $|$ Cross-cultural Knowledge Exchange $|$ Earth Circumference Measurement Expeditions

\end{document}